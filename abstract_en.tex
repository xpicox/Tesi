%!TEX root = thesis.tex

Transactional memory (TM) has emerged as a promising high-level concurrency control mechanism alternative to fine grained lock-based synchronization.
However, most TM models admit only \emph{isolated} transactions, which are not adequate in multi-threaded programming where transactions have to interact via shared data \emph{before} committing.
In this thesis, we present \emph{Open Transactional Memory} (OTM), a programming model supporting \emph{safe, data-driven} interactions between \emph{composable} memory transactions.
In this model, different transactions are transparently \emph{merged} at runtime as soon as they access to shared variables; their threads can then cooperate, until they all either commit or abort together.
Thus, this model relaxes the isolation requirement still guaranteeing atomicity; moreover, it allows for \emph{loosely-coupled} interactions since transaction merging is dynamic and driven only by accesses to shared data, with no need to specify participants beforehand.

We present OTM in the setting of the Haskell language, taking advantage of its type system for guaranteeing composability.  
After having defined the programmer's interface (i.e., types, data structures and constructs), we give some examples such as barriers and Petri nets; moreover, we show that OTM is a conservative extension of STM.
Finally, we present an implementation of OTM in Haskell, as an alternative of STM.