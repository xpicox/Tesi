%!TEX root = thesis.tex

\chapter{Conclusions}
\label{chap:conc}

We have presented OTM, a programming model supporting  interactions between composable memory transactions.
This model separates isolated transactions from non-isolated ones, still guaranteeing atomicity; non-isolated transactions can interact by accessing to shared variables.
Consistency is ensured by transparently \emph{merge} interacting transactions at runtime.  
We have showed the versatility and simplicity of OTM by implementing several fundamental communication idioms (such as producer-consumer, futures, barriers, etc.) and some examples which are incompatible with isolation.
Then, we have provided an Haskell implementation of OTM as a library independent from \emph{STM}, that maintain the same interface against both the user of the library and the Haskell Runtime System; in fact, \emph{OTM} is a conservative extension of \emph{STM}.

A first direction for future work concerns the implementation.
One possibility is to add some heuristics to better handle \emph{retry} events.
Currently, a \emph{retry} restarts all threads participating to the transaction; a more efficient implementation would keep track of the \emph{working set} of each thread, and at a \emph{retry} we need to restart only those threads whose working sets have non-empty intersection with that being restarted.
A solution has already been found and the data structures are designed to support such behavior, but the implementation is lacking.
Furthermore, nested transaction must be added to both isolated and non isolated execution environments.
Nested isolated transaction can be easily integrated in ITM because the data structures do not need further modification; much more work would be required to implement nested open transactions.

From the implementation of \citet{Toneguzzo} we moved a little closer to the Haskell RTS, but further optimizations can be achieved by implementing transactions and OTVars directly in the runtime, along the lines of the implementation of \emph{STM} in the Glasgow Haskell Compiler.

We have presented OTM within Haskell (especially to leverage its type system), but this model is general and can be applied to other STM implementations. A possible future work is to port this model to an imperative object oriented language, such as Java or C++; like other TM implementations, we expect that this extension will require some changes in the compiler and/or the runtime. 