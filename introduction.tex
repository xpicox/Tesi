%!TEX root = thesis.tex

\chapter{Introduction}

\begin{epigraph}{Abraham Lincoln}
Tanto va la gatta al lardo che neanche ci compila\ldots
\end{epigraph}

\section{Lorem Ipsum}
\label{section1}
\lipsum[1-5]

\section{Lorem Ipsum}
\label{section2}
\lipsum[1-5]

\section{Lorem Ipsum}
\label{section3}
\lipsum[1-5]

\section{Lorem Ipsum}
\label{section4}
\lipsum[1-5]

\section{Lorem Ipsum}
\label{section5}
\lipsum[1-5]

\section{Sit Amet}
As seen in \cref{section2,section1,section3,section5}, blah blah.
As proved by \citet{MiculanPT15}, I like 
\textcolor{red}{transactions} \index{transactions} are 
{\tiny Ciao} come va?
\emphidx{beautiful}.

\begin{theorem}
Sia un triangolo rettangolo..
\end{theorem}
\begin{proof}
Vai alle elementari...
\end{proof}

\begin{listing}

\begin{lstlisting}[language=c]
int main() {
  return 0; // ciao
}
\end{lstlisting}
\caption{Main}
\label{lst:main}
\end{listing}


We can define functions, \eg $\sin(x)$, with $x\in\R$ and the 2\nd element 
is.. And \SAT\ is \NP\complete, ma non mi ricordo chi l'ha 
detto\citationeeded.\footnote{LoL}

\fakesection{Altra sezione}



