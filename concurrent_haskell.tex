%!TEX root = thesis.tex

\chapter{Haskell and Concurrency}

Haskell is a functional programming language, whose syntax and semantics are defined in the "Haskell Report" of which the latest revision is Haskell 2010 \cite{Marlow_haskell2010}. Haskell was created by several academic researchers interested in functional languages, to address the lack of of a common language that could be used as a focus for their research.

Two features make it stand out amongst the programming languages crowd:
\begin{itemize}
\item It is \emph{purely functional}, \ie functions can not have side effects or mutate data; for a given input a function always gives the same result.
\item It is \emph{lazy}. Unlike the majority of the programming languages, Haskell does not use strict evaluation: an evaluation strategy in which the arguments of a function are evaluated before the function is called. In Haskell the arguments to a function are passed \emph{unevaluated}, and only evaluated on demand.
\end{itemize}

Next: IO does not fit the functional paradigm because no side effects are permitted and IO requires strictness and laziness gives deliberately unspecified order of evaluation. \cite{PeytonJones:1993:IFP:158511.158524} \todo{IO newtype, token state RealWord \url{https://www.fpcomplete.com/blog/2015/02/primitive-haskell}}
% \url{https://wiki.haskell.org/IO_inside#Haskell_is_a_pure_language}
% pure lazy functional language:
% - side effects only inside monads
% - References to mutable cells


% Haskell was born as pure lazy functional programming language; side effects are 


% Haskell is a lazy functional programming language in which functions with side effects are enclosed inside monads. For instance, I/O actions have type \emph{IO a} and can be combined together by the monadic bind combinator \textgreater \textgreater =. Inside the IO monad can leave also mutable cells. A cell holding values of type a has type IORef a and may be interacted with only via the following operations:

% \begin{lstlisting}
% newIORef :: a -> IO (IORef a)
% readIORed :: IORef a -> IO a
% writeIORef :: IORef a -> a -> IO ()
% \end{lstlisting}

% Concurrent Haskell is the collective name for the facilities that Haskell provides for programming with \emph{internal concurrency}.
% It extends the Haskell 98 Language Report adding explicit concurrency primitives to the IO Monad. 


% This primitives are:
% \begin{lstlisting}
% forkIO :: IO a -> IO ThreadId
% \end{lstlisting}
% that takes an IO action as an argument and executes that action in a new concurrent thread.
