%!TEX root = thesis.tex

\chapter{Haskell and Concurrency}

Haskell is a functional programming language, whose syntax and semantics are defined in the ``Haskell Report'' of which the latest revision is Haskell 2010 \cite{Marlow_haskell2010}. Haskell was created by several academic researchers interested in functional languages, to address the lack of a common language that could be used as a focus for their research.

Two features make it stand out amongst the programming languages crowd:
\begin{itemize}
\item It is \emph{purely functional}, \ie functions can not have side effects or mutate data; for a given input a function always gives the same result.
\item It is \emph{lazy}. Unlike the majority of the programming languages, Haskell does not use strict evaluation: an evaluation strategy in which the arguments of a function are evaluated before the function is called. In Haskell the arguments to a function are passed \emph{unevaluated}, and only evaluated on demand.
\end{itemize}

I/O operations do not fit the functional paradigm because side effects are not permitted and laziness gives deliberately unspecified order of evaluation.

In Haskell, reconcile I/O operations with this features has been achieved by means the IO monad.\cite{PeytonJones:1993:IFP:158511.158524}
\todo{A monad is \url{https://en.wikipedia.org/wiki/Monad\_(functional\_programming)\#Formal\_definition} data constructor, bind and return. Monad laws.
A type to be defined a \emph{monad} must have the following components:
\begin{itemize}
\item A data constructor.
\item A unit function that injects a type in the corresponding monadic type.
\item A binding operation to combine two monadic actions.
\end{itemize}
}
\todo{IO newtype, token state RealWord \url{https://www.fpcomplete.com/blog/2015/02/primitive-haskell}}
As shown in \cref{}, a value of type \emph{IO a} is a monadic action implemented as a function which takes as its input a value representing the entire current state of the world and returns a pair consisting of a value representing the new state of the world, and the result of type \emph{a}.
This introduces \emph{data dependencies} between monadic actions, because each action depends on the \emph{RealWord} value computed by the previous one, hence, preventing the compiler from reordering actions, \emph{sequencing} is introduced for I/O operations.
Besides I/O operations, values of IO include operations with side effects on mutable data types, \eg, a mutable variable ha type \emph{IORef a} and may be accessed only via the following operations:

\begin{lstlisting}
newIORef   :: a -> IO (IORef a)
readIORef  :: IORef a -> IO a
writeIORef :: IORef a -> a -> IO ()
\end{lstlisting}

\section{Concurrent Haskell}
Concurrent Haskell is the collective name for the facilities that Haskell provides for programming with \emph{internal concurrency}.
It extends the Haskell Language Report adding explicit concurrency primitives to the IO Monad.
Haskell threads are created by \emph{forkIO} that takes as argument the IO computation to be performed independently from the main execution path.
\begin{lstlisting}
forkIO   :: IO () -> IO ThreadId
\end{lstlisting}

Thread safe communication and synchronization is achieved through \emph{MVars}.
A value of type \emph{MVar a} is a mutable location that is either empty or full with a value of type \emph{a}.
Interaction with \emph{MVars} is possible only with the primitives:
\begin{lstlisting}
takeMVar :: MVar a -> IO a
putMVar  :: MVar a -> a -> IO ()
\end{lstlisting}
The first empties a full \emph{MVar}, if it is empty the thread waits until is filled. The second fills an empty \emph{MVar} and blocks the thread if it is full. \emph{MVars} can be though as one place channels, or as binary semaphores for thread synchronization.

\section{Haskell STM}

Haskell STM is part of Concurrent Haskell, and adds \emph{transactional actions} that operate on \emph{transactional memory} for safe thread communication. Transactional memory locations are called \emph{transactional variables} or \emph{TVars}.
Transactional actions have type \emph{STM a} and are combined sequentially with the bind operator.
An STM action is provisional during its entire execution and its effects are exposed to the rest of the system by 
\begin{lstlisting}
atomically :: STM a -> IO a
\end{lstlisting}
which takes an STM action and delivers an I/O action that, when performed, runs the transaction guaranteeing atomicity and isolation with respect to the rest of the system.
The type system ensure that IO actions can not be performed inside a transaction, so no irreversible side effects can occur inside the STM monad.

\emph{Transactional variables} have type \emph{TVar a}, where \emph{a} is the type of the value they contain. \emph{TVars} are mutable variables so, like IORefs, they are manipulated only via the interface:
\begin{lstlisting}
newTVar   :: a -> STM (TVar a)
readTVar  :: TVar a -> STM a
writeTVar :: TVar a -> a -> STM ()
\end{lstlisting}

Reads and writes on transactional variables can be combined with the monadic bind operator to define a transactional update:

\begin{lstlisting}
modifyTVar :: TVar a -> (a -> a) -> STM ()
modifyTVar var f = do
    x <- readTVar var
    writeTVar var (f x)
\end{lstlisting}

Then, \emph{atomically (modifyTVar x f)} delivers an IO action that applies \emph{f} to the value held by \emph{x} and updates the variable accordingly as a single atomic isolated operation.

Besides transactional memory interaction, STM allows also for \emph{composable blocking} of transactions with the primitive:

\begin{lstlisting}
retry :: STM a
\end{lstlisting}

The composability property, comes from its type that allows the primitive to be used wherever an STM action may occur.
The semantic of \emph{retry} is to abort the transaction and re-run it after at least one of the transactional variables it has read from has been updated.
As shown in \cite{Harris:2005:CMT:1065944.1065952}, this blocking primitive is enough to implement \emph{MVars} using STM: a value of type \emph{MVar a} is a transactional variable holding a value of type \emph{Maybe a}, i.e. a type that is either \emph{Nothing} or actually holds something of type \emph{a}. A thread applying \emph{takeMVar} to an empty \emph{MVar} is effectively blocked since it retries the transaction upon reading \emph{Nothing} and then it is not rescheduled until the content of the transactional variable changes:

\begin{lstlisting}
type MVar a = TVar (Maybe a)
takeMVar v = do
    m <- readTVar v
    case m of
        Nothing -> retry
        Just r -> writeTVar m Nothing >> return r
\end{lstlisting}

Besides the bind operator, that sequentially combines transactions, the primitive:
\begin{lstlisting}
orElse :: STM a -> STM a -> STM a
\end{lstlisting}
combines transactions as \emph{alternatives}. The \emph{orElse} executes the first transaction, if it retries then is abandoned with no effect and the second is started. If the second retries too, the entire call retries and waits on the variables read by either of the two nested transactions.  
\todo{Esempio: filosofi?}
%In \cref{} is shown the interface of STM. Side effects of transactions on the transactional memory must be delivered through the \emph{atomically} function. A blocking mechanism is introduces 