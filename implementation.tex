%!TEX root = thesis.tex

\chapter{Implementation}

At an high level the implementation is subdivided in an Haskell library and a C library.
The data structures, as well as the operations on them, are defined in the C Header file OTM.h and made available to the Haskell code through the Foreign Function Interface.
Since the FFI does not provide a type safe marshaling for pointers, we decided to introduce the C2HS preprocessor to our build scheme.
All the foreign declarations reside in the file Internals.chs and are written with the syntax of C2HS.\cite{Chakravarty2000}
Finally the OTM interface is exposed by the the Haskell module Control.Monad.OTM defined in the file OTM.hs.

\section{Monads TM, OTM and ITM}
The \emph{TM Monad} is the hearth of our transactional memory implementation; monadic actions of type \emph{TM a} represent transactions that when performed will return a value of type \emph{a}. 
The type \emph{TM a} is a monadic action that represent a transaction.
If wrapped inside the OTM monad the transaction will work on the shared memory otherwise, if wrapped inside the ITM monad
TM is a type synonym for a stack of monad transformers: at the lowest level of the stack, there is the IO Monad that allows to call foreign functions; on top of it, there is the ExceptT monad transformer that adds exception handling to the computations; at top of the transformers stack there is a StateT transformer that adds a mutable state to the underlying monads.
