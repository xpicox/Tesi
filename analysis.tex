%!TEX root = thesis.tex

\chapter{GHC and STM}
\label{chap:ghc}

In this chapter we present an overview of the Haskell compiler and the runtime system.
The first implementation, proposed by \citet{Toneguzzo}, made a heavy use of Haskell STM: all the operations performed on shared data structures were executed by STM transactions. Since this work aims to be an alternative to STM, we decided to set up a new implementation that does not have the STM as a dependency. This led us to thoroughly study STM and the GHC internals.


\section{The Glasgow Haskell Compiler}
Although there are several Haskell compilers, the Glasgow Haskell Compiler (GHC) is the reference compiler of Haskell. GHC was started as part of an academic research project with several goals in mind, one of which was to provide a modular foundation that researchers can extend and develop.

At the highest level GHC can be divided into two parts:
\begin{itemize}
 \item The compiler itself. This is an Haskell program whose job is to convert Haskell source code into executable machine code.
 \item The Runtime System (RTS). This is a large library of C code that handles all the tasks associated with running the compiled Haskell code, including garbage collection, thread scheduling, profiling, exception handling. The RTS is linked into every compiled Haskell program.
 \end{itemize}
The compilation pipeline of a source file involves several phases, with the output of each phase becoming the input of the subsequent phase. What distinguishes GHC from other compilers, and is worth mentioning, is that the source code is translated into several intermediate languages as shown in \cref{fig:pipeline}: in a first phase the source code is parsed and type-checked, then all the syntactic sugar, \ie syntactic constructs that can be translated into simpler constructs, is removed, translating Haskell syntax into a much smaller language called \emph{Core}. The next phase is the optimization one: there are about six passes of optimization each of which takes \emph{Core} and produces \emph{Core}; in this phase, optimizations like dead code elimination, constants propagation or function inlining are applied to the program. Once the \emph{Core} program has been optimized, the process of code generation begins. The code generator first converts the \emph{Core} into a language called \emph{STG}, which is essentially just \emph{Core} annotated with more information required by the code generator. Then, \emph{STG} is translated to \emph{Cmm}, a low-level imperative language with an explicit stack. Finally, the \emph{Cmm} is converted to LLVM code (IR) and passed to the LLVM compiler that produces the native code.

\begin{figure}
\begin{center}
\begin{tikzpicture}[
    auto,>=stealth,
    box/.style={
      draw,
      rectangle,
      inner sep=3pt,outer sep=2pt
    },
    arr/.style={
      double,
      -implies,
    },
    lbl/.style={
      text width=2.5cm,
      align=center,
      yshift=1.5ex
    }
  ]
  \def\d{1.5cm}
  \node[box]                 (n1) {Haskell};
  \node[box, right=\d of n1] (n2) {Core};
  \node[box, right=\d of n2] (n3) {Core};
  \node[box, right=\d of n3] (n4) {STG};
  \node[box, right=\d of n4] (n5) {Cmm};
  \node[box, right=\d of n5] (n6) {LLVM IR};

  \draw[arr] (n1) to node[lbl] {Parsing\\Typechecking\\Desugaring} (n2);
  \draw[arr] (n2) to node[lbl] {Optimization}     (n3);
  \draw[arr] (n3) to                              (n4);
  \draw[arr] (n4) to node[lbl] {Code\\Generation} (n5);
  \draw[arr] (n5) to node[lbl] {Lowering}         (n6);
\end{tikzpicture}
\end{center}
\caption{Structure of GHC compilation pipeline}
\label{fig:pipeline}
\end{figure}


\section{The Spineless Tagless G-Machine}

The Spineless Tagless G-Machine (STGM) is an abstract machine designed to support non-strict higher-order functional languages; this abstract machine defines a precise evaluation model for Haskell code and lays down the layout of Haskell objects involved in the evaluation process. \cite{export:67083}

The Runtime System implements the mapping of the STG language onto an architecture with registers and explicit stack management. It provides the support infrastructure needed for running compiled Haskell code, including a storage manager, responsible for the allocation and the garbage collection, and a scheduler.

The STG-Machine adopts a uniform representation for all heap-allocated objects (whether values or thunks, \ie unevaluated object) and in the Runtime System they are referred to as \emph{Closures}.

Every \emph{Closure} is a contiguous block of memory, consisting of a fixed-format \emph{header} and a \emph{payload} with variable size.
The most important part of the header is the \emph{info pointer},  which points to the info table for the closure.

The \emph{info table} contains all the information that the runtime needs to know about the closure.
Info tables are contiguous blocks of memory that, in addition to provide informations for the storage manager, contain the type of the closure and the \emph{entry code}, \ie the machine code that will evaluate the closure.

Heap objects can be classified in:
\begin{itemize}
\item \emph{executive objects} that participate directly in the execution of a program, \eg thunks, functions, data constructors. They can be subdivided into two kinds of objects based on their type:
\begin{itemize}
\item \emph{Boxed} types are represented by a pointer to an object in the heap;
\item \emph{Unboxed} types are represented by their value, not a pointer.
\end{itemize}
\item \emph{administrative objects} that do not represent values in the original program.
\end{itemize}

The \emph{executive objects} are the citizens of an Haskell program. Among \emph{executive objects} there are:
\begin{itemize}
\item Constructors: used to represent data constructors, whose payload are the fields of the constructor.
\item Primitive objects: used to represent mutable objects such as mutable arrays, mutable variables, MVars and TVars. Their payload varies according to the kind of object.
\item Function closures: used to represent functions. Their payload (if any) consists of the free variables of the function.
\item Thunks: used to represent unevaluated expressions which will be updated with their result. The entry code for a thunk starts by pushing an \emph{update frame} onto the stack. When evaluation of the thunk completes, the update frame will cause the thunk to be overwritten again with an \emph{indirection} to the result of the thunk.
\item Partial applications: used to represent the application of a function to an insufficient number of arguments. Their payload consists of the function and the arguments received so far.
\end{itemize}

\section{The Scheduler}
The  \emphidx{Scheduler} is the heart of the Runtime System and introduces a layer of abstraction over the Operating System.
This layer is build with the assumption that the OS provides some kind of native threads, and for SMP parallelism it is assumed that the OS will schedule multiple OS threads across the available CPUs.

The RTS provides a platform-independent abstraction layer for OS threads, \ie kernel space threads. The Runtime uses the OS threads only for two reasons:
\begin{itemize}
\item To support non-blocking foreign calls, \ie calls to functions defined in an another language
\item To support SMP parallelism
\end{itemize}

When running on an SMP architecture, the RTS starts by creating the number of OS threads specified by the compiler flag +RTS -N, although during the course of running the program more OS threads might be created in order to continue running Haskell code while foreign calls execute.

For each OS Thread known to the runtime, a \emphidx{Task} structure is created. A \emphidx{Task} is a further layer of abstraction over an OS thread; it contains not only a reference to the associated OS Thread, but also a mutex and a condition variable used when OS threads in the runtime need to synchronize with each other or sleep waiting for a condition to occur.

Haskell threads, the ones provided by Concurrent Haskell, are lightweight threads represented by a \emphidx{Thread State Object} (TSO), an ordinary heap objects that contains the complete state of a thread and its stack.
Lightweight threads, otherwise known as green threads or user-space threads,
are a well-known technique for avoiding the overhead of operating system threads: managing threads in user space is cheaper, because fewer traps into the operating system are required, letting the scheduler to manage thousands of threads.

Finally the scheduler has its own representation of a CPU, called a \emphidx{Capability}. The number of \emph{Capabilities} should be chosen, with the option +RTS -N, to be the same as the number of real CPU cores, so that the scheduler never tries to run more Haskell threads simultaneously than the real CPUs available.
The main components of a Capability are:
\begin{itemize}
\item The registers of the virtual machine;
\item The \emph{Task} that is currently animating this Capability;
\item A queue of runnable Haskell threads, called the run queue.
\end{itemize}

\section{Software Transactional Memory}

Among the features offered by the RTS, there is also the support for GHC's Software Transactional Memory implementation.
This implementation is split into two layers.
The top layer implements the STM operations described in \cref{sec:stm}.
This is built on top of the lower layer, which comprises a C library for performing memory transactions that is integrated in the Haskell runtime system.
The lower layer functionalities are made available to Haskell code through the \emphidx{Primitive Operations} (PrimOps), \ie functions that, in the case of STM, are implemented by handwritten Cmm code.
As it will be shown, STM takes full advantage from the explicit stack management offered by the Cmm language.

\subsection{STM implementation}
While executing a memory transaction, a thread-local \emph{transaction log} is built up recording the reads and tentative writes that the transaction has performed.
This transaction log, in the implementation referred to as \emph{transactional record}, is held in a heap allocated object called a \emphidx{TRec} that is pointed to by the TSO of the thread engaged in the transaction.

The log contains an entry for each of the TVars that the memory transaction has accessed.
Each entry, represented by the structure stgTRecEntry, contains a reference to the \emph{TVar} involved, the \emph{old value} held in the TVar when it was first accessed in the transaction, and the \emph{new value} to be stored in the TVar if the transaction commits.
These two values are identical in the case a TVar has been read but not written by the transaction.

Within a memory transaction, all TVar accesses are performed by the primitive operations readTVarzh and writeTVarzh. These accesses remain buffered within the thread’s log, and hence invisible to other threads, until the transaction commits: writes are made to the log, and reads first consult the log so that they see preceding writes from the same transaction.

\subsection{Validation and commit}

The \emph{atomically} (stg\_atomicallyzh PrimOp) function operates by pushing an \emphidx{AtomicallyFrame} entry onto the Haskell execution stack, and invoking stmStartTransaction to allocate a fresh transactional record.
When execution returns to the \emph{AtomicallyFrame}, the log is validated to check that it reflects a consistent view of memory.
For each log entry, validation checks that the old value is pointer-equal to the current contents of the TVar.
If validation succeeds, stmCommitTransaction proceeds applying the changes to the heap. Otherwise, the TRec is discarded, a fresh transaction is started and the atomic block re-executed.
This entire validate-and-then-commit sequence is carried out atomically with respect to all other threads with an incremental lock of the TVars belonging to the write set of the transaction in the validation phase (see validate\_and\_acquire\_ownership STM.c).
Since the \emph{AtomicallyFrame} is a closure, validation and commit are performed by the entry code of the frame's info table.

The mechanism of nested transactions works in a similar way: the \emph{orElse} function pushes onto the stack a \emphidx{CatchRetryFrame}, a closure whose entry code will perform the validation of the nested transactions.

The \emph{retry} PrimOp unwinds the stack in search of an STM's frame and depending on its type will perform different actions:
\begin{itemize}
\item
In the case of an \emph{AtomicallyFrame}, stmWait is called.
The validation procedure is performed to check that the transaction log has seen a consistent view of the heap, and if not the transaction is re-run.
In the consistent case, it allocates new \emph{wait-queue entries}, held in  lists attached to the TVars that the transaction has read.
Once this is done, the calling thread is responsible for blocking itself and reentering the scheduler. The wait queue entries are noticed by a commit which updates the TVars: the updater unblocks any waiters it encounters.
\item
If the frame is a \emph{CatchRetryFrame} and the \emph{retry} is called by the first transaction of the \emph{orElse} then the TRec is discarded and the second transaction is started. If also the second transaction calls a \emph{retry}, the \emph{retry} is propagated further.
If either alternative completes without retrying then the nested transaction is validated by calling stmCommitNestedTransaction to check that it has seen a consistent view of the heap. Validating a nested transaction, validates also its enclosing transactions: if any of them has become invalid by a concurrent update then the whole atomic block re-executes with a fresh log.  If the nested transaction is valid then the TRecs are merged: if the parent already contains an entry for the TVar involved then only the \emph{new value} is copied from the nested TRec, otherwise the entire entry is copied.
\end{itemize}

\section{Foreign Function Interface}

The Foreign Function Interface (FFI) allows Haskell programs to cooperate with libraries written in C.
Haskell programs can call foreign functions and foreign functions can call Haskell code.
The FFI defines the marshaling for some boxed types such us Int, Double, Char and exports un-boxed foreign types that map directly to C types such as Int64, for signed 64-bit integers, Word64, for unsigned 64-bit integers, Ptr a, for pointers, and FunPtr a, for function pointers.

The syntax for declaring a foreign import allows decide the behavior of the runtime when a call to the C function is made:
\begin{itemize}
\item If the import is marked as \emph{safe}, then the runtime pushes the arguments onto the Haskell stack and returns a C function descriptor to the scheduler.
The scheduler suspends the Haskell thread, spawns a new OS thread which pops the arguments off the Haskell stack onto the C stack, calls the C function, pushes the result onto the Haskell stack and informs the scheduler that the C function has completed and the Haskell thread is now runnable.
This behavior is called \emph{safe} because if the C calls blocks for a long time, the capability from which the call was made can continue to execute Haskell threads on its OS thread.
\item If the import is marked as \emph{unsafe} then the function is executed by the same OS thread that was executing the Haskell code that made the call. An unsafe call is faster, but can block a capability and possibly block the entire Haskell program.
\end{itemize}

Since the majority of foreign functions have side effects, they live inside the IO monad as functions that return a value of type IO a. The foreign import can be marked as \emph{pure} and the function will return a \emph{pure value}: is programmer's duty to ensure that there are no side effects, because, in a pure environment, the strict order execution guarantee is lost.

Finally, the FFI defines two kinds of pointers that rule the memory management of a pointed objects:

\begin{itemize}
\item \emphidx{Foreign Pointers} represent references to objects that are maintained in a foreign language, \ie that are not part of memory managed by the Haskell storage manager.
The essential difference between a foreign pointer and a raw pointer is that the former may be associated with a \emph{finalizer}, \ie, a routine that is invoked when the Haskell storage manager detects that there are no more references left that are pointing to the \emph{Foreign Pointer}.
So the \emph{finalizer} is responsible to free the resource bound to the foreign pointer.

\item \emphidx{Stable Pointers}\label{ffi:stableptr} resolve the dual problem: the life span of an Haskell object is no longer managed by the storage manager, indeed it becomes responsibility of the programmer to release the resource.
The function \emph{newStablePtr} creates a stable pointer referring to the given Haskell value and as long as \emph{freeStablePtr} is not called, that pointer will remain valid.
So a \emph{Stable Pointer} is a reference to a Haskell expression that is guaranteed not to be affected by garbage collection, i.e., it will neither be deallocated nor will the value of the stable pointer itself change during garbage collection. \emph{Stable Pointers} can be passed to foreign code, which can treat it as an opaque reference to a Haskell value.
\end{itemize}